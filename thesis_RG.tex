\documentclass[12pt,a4paper,titlepage,oneside,abstract=true,toc=listof,toc=bibliography]{scrreprt}

\usepackage[utf8]{inputenc}
\usepackage{amsmath}
\usepackage{amsfonts}
\usepackage{amssymb}
\usepackage[titletoc]{appendix}
\usepackage[english]{babel}
\usepackage{caption}
\usepackage{graphicx}
\usepackage[colorlinks=true,urlcolor=blue,citecolor=red,linkcolor=red,bookmarks=true,linktoc=all,bookmarkstype=toc,bookmarksopen=true]{hyperref}
\usepackage{longtable}
\usepackage{natbib}
%\usepackage{biblatex}
\usepackage[intoc,english]{nomencl}
\usepackage{textcomp}
\usepackage{xpatch}
\usepackage{pdfpages}

%zähler ohne kapitelnummern
\counterwithout{figure}{chapter}
\counterwithout{table}{chapter}

\begin{document}

%----------------------------------------------------------------------------------------
%	TITLE PAGE
%----------------------------------------------------------------------------------------

\begin{titlepage} % Suppresses displaying the page number on the title page and the subsequent page counts as page 1
	\newcommand{\HRule}{\rule{\linewidth}{0.5mm}} % Defines a new command for horizontal lines, change thickness here
	
	\center % Centre everything on the page
	
	%------------------------------------------------
	%	Headings
	%------------------------------------------------
	
	\textsc{\LARGE Humboldt University of Berlin}\\[1.5cm] % Main heading such as the name of your university/college
	
	\textsc{\Large Institute for Library and Information Science}\\[0.5cm] % Major heading such as course name
	
	%\textsc{\large Minor Heading}\\[0.5cm] % Minor heading such as course title
	
	%------------------------------------------------
	%	Title
	%------------------------------------------------
	
	%\HRule\\[1cm]
	\vfill
	{\large\textbf{Seeking Research Software. A Qualitative Study of Humanities Scholars' Information Practices.}}\\[0.4cm] % Title of your document
	
	%\HRule\\[3.6cm]
	\vfill	
	
	%------------------------------------------------
	%	Author(s)
	%------------------------------------------------
	
			\large
			\textsc{Ronny Gey} % Your name
	
	%------------------------------------------------
	%	Date
	%------------------------------------------------
	
	\vfill\vfill\vfill % Position the date 3/4 down the remaining page
	
	{\large\today} % Date, change the \today to a set date if you want to be precise
	
	%------------------------------------------------
	%	Logo
	%------------------------------------------------
	
	\vfill\vfill
	\includegraphics[width=0.2\textwidth]{husiegel_bw_op.eps}\\[1cm] % Include a department/university logo - this will require the graphicx package
	 
	%----------------------------------------------------------------------------------------
	
	\vfill % Push the date up 1/4 of the remaining page
	
\end{titlepage}

%%======================================================================
%%      Kurzfassung / Abstract
%%======================================================================
\def\abstractname{Zusammenfassung}
\begin{abstract}
Zusammenfassung
\end{abstract}

\def\abstractname{Abstract}
\begin{abstract}
Abstract
\end{abstract}

%%======================================================================
%%      InhaltsVZ, AbbildungsVZ, TabellenVZ, AbkürzungsVZ
%%======================================================================
\pagenumbering{roman}
\pdfbookmark{\contentsname}{Contents}
\tableofcontents
\cleardoublepage
\listoffigures                        
\cleardoublepage
\listoftables
\cleardoublepage
\pagenumbering{arabic}
\cleardoublepage
\printnomenclature

%%======================================================================
%%      Introduction
%%======================================================================

\chapter{Introduction}
Today, software is a central component of science. Throughout the entire research life cycle, researchers use software tools for data collection, transformation, analysis and presentation as well as for generating hypotheses, managing literature and writing scientific papers \citep{Kethers2017, Pan2016, Wolski2017}. Software has changed the way we actually do science. The complexity of the analyses carried out by researchers has increased, as has the amount of data that researchers can process. Software supports the documentation of the research process and enables reproducibility \citep{DallmeierTiessen2016, Waltemath2016} and accuracy of results.

Due to the increased importance of research software for research \citep{Katz2017} and the increase in the sheer number of software, it is all the more important for researchers to identify suitable software and select the one which best fits the research problem, the actual step in the research process, or the research data which has to be processed, and, in consequence, which satisfies the researchers information need \citep{Wilson1994}. In addition to increased efforts, difficulties in seeking software can also endanger the scientific reproducibility of studies or even lead to multiple developments of software with identical functions instead of reusing existing software \citep{Hucka2018}.

Information seeking of researchers is generally of great interest within the field of information, be it information behavior \citep[e.g.]{Ahmadianyazdi2018, Barrett2005, Campbell2017, Catalano2013, Ellis1993, Hemminger2007, Korobili2011,  Liyana2017, Rimmer2006, RuppSerrano2013, Wang2008} or information practices \citep[e.g.]{Boyum2015, Bulger2011, Fry2006, Given2018, Roos2015}. However, seeking software is still rather challenging for researchers \citep{Howison2015}. In a recent study, \citet{Hucka2018} surveyed scientists and engineers from several fields to better understand their approaches and selection criteria for seeking software. They found out that "\textit{finding software suitable for a given purpose remains surprisingly difficult}". Even in the domain of software development, the main challenge for software reuse are difficulties in finding software artifacts as \citet{Bauer2014} revealed in a study on code reuse at Google. \citet{Grossman2009} identified users unawareness of specific software tools. These results are all the more surprising because the participants in the cited studies come from a group with a greater affinity for software (software developers, engineers).

The lack of awareness of specific software tools among researchers has been addressed by several technical solutions. Code aggregators, specialized search engines, programming language package repositories, code and application repositories, research repositories (e.g. Zenodo or Figshare), and curated web lists and catalogues aid users in discovering software \citep{Struck2018}. Standards and tools for citing software enable researchers to identify, cite and reuse software \citep[e.g.]{Niemeyer2016, Smith2016, Soito2017}. Research funding agencies and research organizations \citep[e.g.]{Haupt2018, Katerbow2018, Scheliga2019} adopt guidelines for the development and use of research software with the aim of increasing the reusability and quality of the software artifacts developed. In turn, the technical solutions presented are also aimed more at a technically experienced audience, often even at software developers directly. For researchers with less experience in the use of software, e.g. from the humanities \citep{Rimmer2006}, seeking software remains a difficult undertaking.

The information-seeking behavior of humanities scholars in general has been addressed widely \citep[e.g.]{Barrett2005, Bronstein2007, Bronstein2007a, Catalano2013, Ellis1993, Given2018, Korobili2011, Liew2006, Rimmer2006}. In his pioneering work on Grounded Theory in information-seeking, \citet{Ellis1993} identified patterns of information-seeking of social sciences, sciences, and humanities scholars. In 2005, \citet{Barrett2005} analyzed information-seeking habits of graduate student researchers in the humanities. Korobili{2011} examined factors influencing information-seeking behavior at the philosophy faculties. While studies in information behavior draw on the cognitive viewpoint, information practice studies are guided by the ideas of social constructionism and collectivism \citep{Savolainen2007, Talja2005, Talja2007}. Humanities scholars information-seeking practices have also been addressed in several studies \citep{Benardou2013, Bulger2011, Given2018, Palmer2009}. In previous studies, however, the classic research process of humanities scholars has been examined, which is mainly based on literature research. Although the information-seeking in the humanities is also based on software tools, e.g. databases, web-based editions, search engines, or online journals \citep{Barrett2005, Rimmer2006}, the search for software itself is rarely discussed. One of these rare examples, however a non-humanities one, is Hepworths et al. \citeyearpar{Hepworth2017} study of journalism professors' information-seeking behavior. While seeking new online tools, journalism professors rely on other journalism professors, followed closely by media-related foundations, media professionals, and conferences.

%%======================================================================
%%      Theory
%%======================================================================
\chapter{Theory}
%%======================================================================

\section{Information Seeking}
- Information Science briefly described

	Seeking, Searching, Retrieval

- Information Seeking Research (Ingwersen2005)
-- Concepts: Strategies
-- Collaborative IS: Shah2013 

- Distinction between behaviour and practices: 

	the concepts of information behavior and information practice emerge from different discourses that open alternative viewpoints on information seeking. Savolainen2007 

	Bates2010 - information behaviour, Case2007 - information behaviour

- behaviour: wilson, ellis, kuhltau, Niedzwiedzka2003
-- different conceptualizations: intra/inter/extrapersonal (Feinman
-- transgender ib: pohjanen2016

- practices: McKenzie and Talja 

\section{Information Practices}
- Introduction:
	Savolainen2007, Talja2007

The social constructionist paradigm puts emphasis on social practices, ``the concrete and situated activities of interacting people, reproduced in routine social contexts across time and space’’ (Rosenbaum, 1993, p. 239). A focus on
practices rather than on behaviour shifts the analysis from cognitive to social and is consistent with the study of information seekers within their social context (for examples, see Rothbauer (2002), McKenzie and Davies (2002)).

- Starting with McKenzie
	McKenzie2003, 2003a

- and Talja

- further examples of Information Practices: Savolainen 2007 - LitReview

\section{Research Software}
- definition
- examples
- importance for research, in the research process

\subsection{Information Practices towards research software}
- examples of studies, what has been studied yet

\section{Domain Analysis: Humanities/Philology}
- short: humanities, long: philology
- definition
- characteristics: subjects, work procedures, tools, ...

\subsection{Information Practices of Humanists}
- examples of studies, what has been studied yet
- bisher nicht viel gefunden, practices of other scholars, but humanists seldom
	
%%======================================================================
%%      Research Design
%%======================================================================
\chapter{Research Design}
Since "\textit{[u]nderstanding the nature of information practices and their relation to the production of scholarship is important for both theoretical and applied work in library and information science (LIS)}" \citep[p. 165]{Palmer2009} this thesis sets out to study information practices of humanities scholars and their seeking for research software to better understand humanists needs and future LIS services \citep{Case2008, Cunningham2010}. With information practices we mean practices of seeking, managing, giving, and using information in context \citep{Palmer2009}. I chose an exploratory study design \citep{Rinsdorf2013} where the personal realm of experience of each interviewee lies in the center of the analysis. Through an explorative approach to the object of research, qualitative social research approaches the social and subjective reality constructed by humans. Theory development is usually inductive, using the individual cases of empirical studies. Acting in a sociocultural context is always an interpretative process. In this respect, \citet[p. 20ff]{Lamnek2005} distinguishes six principles that qualitative research should be guided by: openness, communication, process-like, reflexivity, explication, and flexibility.

The aim of this work is to investigate the information-seeking practices of early-career philologists when seeking research software (RQ1). This research focuses on philologists problems, contradictions and barriers in finding information and their information sharing about research software. Special emphasis will be placed on the respondents' recourse to their own research process and the knowledge and practice structures in their field \citep{Hjorland1995} which are socially constructed (RQ2):

\begin{tabular}{p{2cm}p{12cm}}
& \\
\textbf{RQ1}: & What information seeking practices humanities scholars engage in when looking for software to use with research data? \\ 
\textbf{RQ2}: & How do domain specific structures shape the information practices of humanists? \\ 
& \\
\end{tabular}\\  

With this thesis I want to comply with the principles of an open science\footnote{\url{https://ocsdnet.org/manifesto/open-science-manifesto/}}. Hence, I have decided that all data generated during the concept, survey, analysis, and writing phase is publicly available on GitHub\footnote{\url{https://github.com/geyslein/Masters_Thesis}}, as long as it meets research ethics standards of the \citet{DeutscheForschungsgemeinschaft2019}: Interview audio records and unanonymized interview transcripts are excluded. Prior to the study, I obtained the consent of the interviewees regarding the interview recording and later regarding the transcribed interviews.

\section{Methods of Data Collection and Processing}
%% Einleitung
According to \citet[p. 329]{Lamnek2005}, the qualitative paradigm sees the interview as the ideal way to collect data. Compared to the participating observation, it is much easier to realize. Moreover, qualitative research on the interpretation of texts is very advanced. The qualitative interview can be recorded, intersubjectively traced, and arbitraryly reproduced. Consequently, the findings of the study are derived from accounts of information practices and not from observation of information practices as it happened. 

%%Sampling
The first step in data collection was to establish contact with the potential interview partners. It was necessary to consider criteria based on preliminary considerations derived from theory. The goal of this work is to investigate information practices of humanities scholars in the search for software. I assumed that software is now an integral part of scientific work in the humanities as well, so I did not make any further preliminary considerations in this regard. Since there is a very large spectrum of subjects within the humanities, I had to minimize the resulting side effects during the investigation. Practical criteria of field access helped with the selection, according to which I chose the area of classical philology. With this convenience sampling strategy, I gained access to two participants with little effort. The two participants then provided referrals for new participants. Through the snowball sampling strategy \citep{Biernacki1981}, I was able to achieve a desired sample size of 5 test persons relatively quickly. None of the contacted persons refused the participation in the study.

%% Interviews
Interviews are the main data gathering technique which are guided by semi-structured interview guidelines, and implemented in a face-to-face manner \citep{Bryman2004} in German language. With the interviews I obtain emotions, thoughts, and intentions of the participants and discover their perspective of the social world \citep{Patton2002}. Due to the corona pandemic situation, I originally planned the interviews as a virtual setup with the help of web conferencing software. Since a reduced attention span and lack of experience with web conferencing was assumed among the interviewees, I planned two interviews per participant. The first two interviewees already accepted this setting. In the meantime, however, the first relaxation measures took effect, which is why we agreed on face-to-face interviews. Such a setup with two staggered interviews (planned minimum interval of 2 weeks) increases flexibility, as it allows for adjustments in the second interview after the analysis of the first interview. In addition, I was able to refer to contents of the first interview. Furthermore, it enabled reflection and self-observation processes among the interviewees regarding the search and use of software. I conducted 9 interviews. Four interviewees were interviewed twice. The interview duration varied between 60 and 90 minutes. Due to practical reasons, the fifth interviewee agreed to conduct a longer interview (duration 125 minutes) instead of the two-interview-setup.

The interviewees agreed to the interviews, their recording and the publication of the anonymized protocols by signing a declaration of consent at the beginning of the first interview\footnote{see \url{https://github.com/geyslein/Masters_Thesis/blob/master/_data/collection/consent\%20form\%20interview\%20(German).md} for the consent form}. In the beginning of the interviews, I briefly introduced myself and the background of the study. All interviews were conducted in the offices of the interviewees or in meeting rooms of the institute. During the interviews I followed the structure of the interview guidelines.

% Interviewleitfaden
The interview guidelines support the thematic structuring of the interview, but should nevertheless leave enough free space for the qualitative process during the interview as well as for the ideas of the interview partners. At the beginning of the interview guidelines, there are questions about studies, doctorate, previous work experience and the current position. It is followed by questions about the contents, the methods used and the theories of the own research. Afterwards, the field of investigation is approached by discussing the search process in general, using the example of one's own literature research. And on the other hand, the not very familiar topic of software will be opened up by means of easy introductory questions (Section A).  Section B focuses on the sources that are used in the search for software. The largest of the sections, Section C, focuses on information practices in the search for and use of software. Based on the literature on information practices, the section is divided into Seeking, Scanning, Monitoring, Proxy, Context and Avoiding. In Section D, the last part of the interview guide, follow-up questions for the second interview are provided, which are chosen depending on the respondent and the course of the first interview. The complete interview guide is listed in the appendix (\ref{sec:Interview_guide}).

% Aufnahme
To ensure the reproducibility of the interviews, they were recorded with a standard voice recorder and, in parallel, with a voice recording app for an android smartphone. The interviews were recorded without disturbances as far as possible and could be transcribed completely due to good recording quality. The recording of an interview actually contradicts the demand for a natural communication situation and possibly causes rejection on the part of the interviewees. Guaranteeing anonymity and informing the interviewees about the necessity of recording should prevent this. By using tape or digital recording devices, it is also possible to conduct the interview situation inconspicuously and in its natural environment.

%% Transcription
For a later analysis of the conversation content a transcription is necessary. Transcription means the writing of audiovisual material, in this case the audio recordings of the interviews. The transcript as a result of this process thus already embodies a first subjective selection and reduction of the interview recording \citep[p. 321]{Edwards2003}. Once the interviews were conducted, I orthografically transcribed the interviews recordings using EasyTranscript, an open source transcription software\footnote{\url{https://www.e-werkzeug.eu/index.php/de/produkte/easytranscript}}. The transcription system used for this purpose should be based on the specific research objective \citep[p. 331]{Edwards2003}. The aim of this work is an evaluation of the discussions in terms of content and topic. A sophisticated transcription system, which is required for a conversation-analytical examination, was therefore not necessary. I chose a verbatim and partially annotated transcription system, which considers several para- and non-verbal aspects. The following conventions were applied:

\small
\begin{table}
\caption{Transcription conventions}
\centering
\begin{tabular}{ll}
 & \\
\hline
\hline
 	(lachen) & Laughter\\
 	ähm	& Non-lexical utterance (uh, erm, um)\\
 	{[Anony]} & Anonymized parts of the transcript\\
	(DESCRIPTIONS)	& Further Explanations if text was anonymized\\
	               & or not considered (off-topic)\\
\hline
 & \\
\end{tabular}
\end{table}
\normalsize

I did not check the finished transcripts for correct spelling after the transcription for reasons of time economy. Further, I translated the relevant text passages for the final thesis into English anyway, whereby attention was paid to the correct use of spelling and grammar.

%% Anonymisierung UND Cross-Check probanden
In the next step, the interview transcripts were anonymized from information regarding person, institutes, university, thesis and research topics, and criticism expressed during the interview. This was done at the request of all interview participants. They attached great importance to this step already before we conducted the interviews. Further, they intensively controlled the transcripts after the anonymization. On request, I then anonymized further aspects in the interviews.

%% Statistik Transkripte
The process finally results 12 hours and 22 minutes of audio recordings and 162 pages of interview transcripts (Arial, 12pt). 

\section{Process of Data Analysis}
After I received consent from all participants for the anonymized research transcripts, I analyzed the transcripts with a qualitative content analysis to explore qualified hypotheses \citep{Kohlbacher2006, Krippendorff2012, Mayring2000, Mayring2014}. It enables the researcher to include textual information and to identify its properties systematically. In detail, I chose a qualitative content analysis according to \citet{Mayring2014}. For the analysis process I used an open source software called QualCoder\footnote{\url{https://github.com/ccbogel/QualCoder}}, which supports coding, annotation and category building. 

% open coding
In a first step, I approached the material using systematic open coding \citep{Corbin1990} to conceptualize and categorize the interview data. For this purpose, I have deductively identified categories from the knowledge about information practices, the field, and the process of compiling the interview guide. During the analysis, I inductively derived codes and categories and constantly revised the existing ones. The previously established category system is thus constantly modified and further developed, taking into account the demand for openness and flexibility of the research process. 

% inhaltsanalyse - eventuell noch category system näher beschreiben
\citet[p. 65]{Mayring2014} describes three basic interpretation techniques: reduction, explication and structuring. With the analysis I aim to \textit{reduce} the material to the core statements. For the interpretation of the present interview data I have chosen the \textit{inductive category formation} as a form of data reduction. In contrast to summarizing, the other and very extensive reduction method, inductive category formation considers only those parts relevant for the research question and the step of paraphrasing is skipped \cite[p. 79]{Mayring2014}. The final category system can be looked up in the Appendix (\ref{sec:Category_system}).
	
%%======================================================================
%%      Findings
%%======================================================================
	
\chapter{Findings}

\section{Introduction of the Empirical Field}
% the institute, small, family like, history
Classical Philology is one of the oldest subjects at the university. It first flourished in the sixteenth century. In the nineteenth century, the institute was a center of classical philology. The dominant field of work in the nineteenth century was edition philology. In the twentieth century, scholars opened up the great texts of antiquity in commentaries and interpretations. Similar to classical philology, the institute has declined in importance since its zenith in the 19th century and today counts about 20 scholars. Both Latin and Greek studies can be studied at the institute.
% interviewees
In the present study I interviewed 5 members of the institute who are engaged as research assistants, PhD students or in post-doc positions at the institute. A brief profile of each interviewee is depicted in table \ref{tab:interviewees}. 

\small
\begin{table}
\caption{Overview of interviewees}
\label{tab:interviewees}
\centering
\begin{tabular}{llll}
& & & \\
\hline
Name & Position & No. Interviews & Duration of Interviews\\
\hline
\hline
Sandra & Post-doc & 2 & 2h25min\\
Peter & PhD student & 1 & 2h09min\\
Marie & Post-doc & 2 & 2h44min\\
Marco & PhD student & 2 & 2h56min\\
Christian & PhD student & 2 & 2h07min\\
\hline
& & & \\
\end{tabular}
\end{table}
\normalsize

% field kP
%% Eigenschaften
During the interviews the participants attributed various characteristics to classical philology or philologists that they perceived as typical. %conservative 
In Sandra's opinion, classical philologist "\textit{[...] are quite conservative in their approach. They work the way they have always worked.} [Sandra, 1st interview\footnote{In the following, I use the convention [NAME, No. of Interview], for example [Sandra, 1] for the first and [Sandra, 2] for the second interview}]. 
%critical




\begin{quotation}
\textit{ein teilbereich usnerer arbeit nennt sich ja auch quellenkritik und das ist eine ganz zentrale frage, also ähm ja und die lässt sich auch in verschieden teilbereiche spalten. das eine ist ja die sicherheit oder nichtsicherheit, die man haben kann, was die gestalt des textes selber angeht, der ja wie gesagt, der ja überlieferungsprozesse und zum teil auch transformationsprozesse durchlaufen hat über zwei jahrtausende in aller regel. } [Peter, 1]
\end{quotation}

%%%analog

%%%critical und exakt

%%%it-fern

%%% 
%% Teilbereiche
%% Methoden

\section{Information Practices in Research}
% sources

% 

\section{Software Use}
% wenig

% literaturverwaltung word - citavi

% typegreek

\section{Information Practices Towards Software}
% 
% 

\section{Domain Factors}
% neg erfahrung mit DH projekten

% Die Grenze meiner Sprache... zitat von wittgenstein!

% geringe methodenreflexion

% geringe prio - sw suche früher abbruch

% vertrauen - wenige quellen, antriggern von freunden/kollegen, primärquelle bei schwierigem


%%======================================================================
%%      Discussion
%%======================================================================
\chapter{Discussion}

%%% Hypo 1
- classical philologists do not reflect much on the research process - which is why they do not regarding software use either => as a matter of fact, the analysis focus is on information practices in the widest sense


%%% Hypo 2
- tool selection is very much trust based (recommendations/consulting from colleagues and friends/family)

%%% Hypo 3
- negative experiences in digital humanities projects induce skepticism towards digital tools

\cite{Zundert2012} large scale digi infrastructures as dead end
\cite{Neuefeind2020} - Sustainability Strategies for Digital Humanities Systems

%%% Hypo 4
- difficulties in formulating the right search terms (what they describe as well as what I could listen to during the interviews)
==\cite{Savolainen2015a}: cognitive barriers to information seeking - Inability to articulate one’s information needs; Poor search skills

%%% Maybe Conclusion?
\cite{Constant1996} - kindness of strangers
\cite{Edmond2005} - role of prof intermediaries
\cite{Gunning1978} - librarian should participate in research process
\cite{MonroeGulick2013} - librarians as partners

%%======================================================================
%%      Conclusion
%%======================================================================
\chapter{Conclusion}

%%======================================================================
%%      Zusammenfassung
%%======================================================================
\chapter{Zusammenfassung (German conclusion)}

\bibliographystyle{apalike} 
%\bibliographystyle{plainnat} % use this to have URLs listed in References
\cleardoublepage			
\bibliography{../../Literatur/UB} % Path to your References.bib file		

%%======================================================================
%%	Selbstständigkeitserklärung
%%======================================================================
%\chapter*{Declaration of independence}
%%Hiermit erkläre ich, dass ich die vorliegende Arbeit selbstständig angefertigt, nicht anderweitig zu Prfüngszwecken vorgelegt und keine anderen als die angegebenen Hilfsmittel verwendet habe. Sämtliche  wissentlich verwendeten Textausschnitte, Zitate oder Inhalte anderer Verfasser wurden ausdrücklich als solche gekennzeichnet.\\[2ex]
%I hereby declare that I have prepared this paper independently, have not submitted it for testing purposes and have not used any other aids than those specified. All knowingly used text excerpts, quotations or contents of other authors have been explicitly marked as such. \\[2ex]
%
%Leipzig,  \today\\[6ex]
%\flushleft
%\newlength\us
%\settowidth{\us}{Ronny Gey}
%\begin{tabular}{p{\us}}\hline
%\centering\footnotesize Ronny Gey
%\end{tabular}

%======================================================================
%	Anhang
%======================================================================
%\cleardoublepage\pdfbookmark[-1]{Anhang}{Anhang}
\cleardoublepage

\begin{appendices}
\appendixpage

%\cleardoublepage

\chapter{Interview Guide}
\label{sec:Interview_guide}
\includepdf[pages=-,pagecommand={\thispagestyle{empty}}]{_methods/Interview_guidelines_en.pdf}
\cleardoublepage

\chapter{Category System}
\label{sec:Category_system}
%\includepdf[pages=-,pagecommand={\thispagestyle{empty}}]{_analysis/....pdf}
\cleardoublepage

%\chapter{Consent Form}
%\includepdf[pages=-,pagecommand={\thispagestyle{empty}}]{_data/collection/consent form interview (German).pdf}

\end{appendices}
%======================================================================
%	Selbstständikeitserklärung
%======================================================================
\includepdf[pages=-,pagecommand={\thispagestyle{empty}}]{_admin/Selbststaendigkeitserklaerung}
\cleardoublepage

\end{document}