\documentclass[12pt,a4paper,titlepage,oneside,abstract=true,toc=listof,toc=bibliography]{scrreprt}

\usepackage[utf8]{inputenc}
\usepackage{amsmath}
\usepackage{amsfonts}
\usepackage{amssymb}
\usepackage[titletoc]{appendix}
\usepackage[english]{babel}
\usepackage{caption}
\usepackage{graphicx}
\usepackage[colorlinks=true,urlcolor=blue,citecolor=red,linkcolor=red,bookmarks=true,linktoc=all,bookmarkstype=toc,bookmarksopen=true]{hyperref}
\usepackage{longtable}
\usepackage{natbib}
\usepackage[intoc,english]{nomencl}
\usepackage{textcomp}
\usepackage{xpatch}

%zähler ohne kapitelnummern
\counterwithout{figure}{chapter}
\counterwithout{table}{chapter}

\begin{document}

%----------------------------------------------------------------------------------------
%	TITLE PAGE
%----------------------------------------------------------------------------------------

\begin{titlepage} % Suppresses displaying the page number on the title page and the subsequent page counts as page 1
	\newcommand{\HRule}{\rule{\linewidth}{0.5mm}} % Defines a new command for horizontal lines, change thickness here
	
	\center % Centre everything on the page
	
	%------------------------------------------------
	%	Headings
	%------------------------------------------------
	
	\textsc{\LARGE Humboldt University of Berlin}\\[1.5cm] % Main heading such as the name of your university/college
	
	\textsc{\Large Institute for Library and Information Science}\\[0.5cm] % Major heading such as course name
	
	%\textsc{\large Minor Heading}\\[0.5cm] % Minor heading such as course title
	
	%------------------------------------------------
	%	Title
	%------------------------------------------------
	
	%\HRule\\[1cm]
	\vfill
	{\large\textbf{Seeking Research Software. A Qualitative Study of Humanities Scholars' Information Practices.}}\\[0.4cm] % Title of your document
	
	%\HRule\\[3.6cm]
	\vfill	
	
	%------------------------------------------------
	%	Author(s)
	%------------------------------------------------
	
			\large
			\textsc{Ronny Gey} % Your name
	
	%------------------------------------------------
	%	Date
	%------------------------------------------------
	
	\vfill\vfill\vfill % Position the date 3/4 down the remaining page
	
	{\large\today} % Date, change the \today to a set date if you want to be precise
	
	%------------------------------------------------
	%	Logo
	%------------------------------------------------
	
	\vfill\vfill
	\includegraphics[width=0.2\textwidth]{husiegel_bw_op.eps}\\[1cm] % Include a department/university logo - this will require the graphicx package
	 
	%----------------------------------------------------------------------------------------
	
	\vfill % Push the date up 1/4 of the remaining page
	
\end{titlepage}

%%======================================================================
%%      Kurzfassung / Abstract
%%======================================================================
\def\abstractname{Zusammenfassung}
\begin{abstract}
Zusammenfassung
\end{abstract}

\def\abstractname{Abstract}
\begin{abstract}
Abstract
\end{abstract}

%%======================================================================
%%      InhaltsVZ, AbbildungsVZ, TabellenVZ, AbkürzungsVZ
%%======================================================================
\pagenumbering{roman}
\pdfbookmark{\contentsname}{Contents}
\tableofcontents
\cleardoublepage
\listoffigures                        
\cleardoublepage
\listoftables
\cleardoublepage
\pagenumbering{arabic}
\cleardoublepage
\printnomenclature

%%======================================================================
%%      Introduction
%%======================================================================

\chapter{Introduction}
Today, software is a central component of science. Throughout the entire research life cycle, researchers use software tools for data collection, transformation, analysis and presentation as well as for generating hypotheses, managing literature and writing scientific papers \citep{Kethers2017, Pan2016, Wolski2017}. Software has changed the way we actually do science. The complexity of the analyses carried out by researchers has increased, as has the amount of data that researchers can process. Software supports the documentation of the research process and enables reproducibility \citep{Dallmeier-Tiessen2016, Waltemath2016} and accuracy of results.

Due to the increased importance of research software for research \citep{Katz2017} and the increase in the sheer number of software, it is all the more important for researchers to identify suitable software and select the one which best fits the research problem, the actual step in the research process, or the research data which has to be processed, and, in consequence, which satisfies the researchers information need \citep{Wilson1994}. In addition to increased efforts, difficulties in seeking software can also endanger the scientific reproducibility of studies or even lead to multiple developments of software with identical functions instead of reusing existing software \citep{Hucka2018}.

Information seeking of researchers is generally of great interest within the field of information, be it information behavior \citep[e.g.]{Ahmadianyazdi2018, Barrett2005, Campbell2017, Catalano2013, Ellis1993, Hemminger2007, Korobili2011,  Liyana2017, Rimmer2006, Rupp-Serrano2013, Wang2008} or information practices \citep[e.g.]{Boyum2015, Bulger2011, Fry2006, Given2018, Roos2015}. However, seeking software is still rather challenging for researchers \citep{Howison2015}. In a recent study, \citet{Hucka2018} surveyed scientists and engineers from several fields to better understand their approaches and selection criteria for seeking software. They found out that "\textit{finding software suitable for a given purpose remains surprisingly difficult}". Even in the domain of software development, the main challenge for software reuse are difficulties in finding software artifacts as \citet{Bauer2014} revealed in a study on code reuse at Google. \citet{Grossman2009} identified users unawareness of specific software tools. These results are all the more surprising because the participants in the cited studies come from a group with a greater affinity for software (software developers, engineers).

The lack of awareness of specific software tools among researchers has been addressed by several technical solutions. Code aggregators, specialized search engines, programming language package repositories, code and application repositories, research repositories (e.g. Zenodo or Figshare), and curated web lists and catalogues aid users in discovering software \citep{Struck2018}. Standards and tools for citing software enable researchers to identify, cite and reuse software \citep[e.g.]{Niemeyer2016, Smith2016, Soito2017}. Research funding agencies and research organizations \citep[e.g.]{Haupt2018, Katerbow2018, Scheliga2019} adopt guidelines for the development and use of research software with the aim of increasing the reusability and quality of the software artifacts developed. In turn, the technical solutions presented are also aimed more at a technically experienced audience, often even at software developers directly. For researchers with less experience in the use of software, e.g. from the humanities \citep{Rimmer2006}, seeking software remains a difficult undertaking.

The information-seeking behavior of humanities scholars in general has been addressed widely \citep[e.g.]{Barrett2005, Bronstein2007, Bronstein2007a, Catalano2013, Ellis1993, Given2018, Korobili2011, Liew2006, Rimmer2006}. In his pioneering work on Grounded Theory in information-seeking, \citet{Ellis1993} identified patterns of information-seeking of social sciences, sciences, and humanities scholars. In 2005, \citet{Barrett2005} analyzed information-seeking habits of graduate student researchers in the humanities. Korobili{2011} examined factors influencing information-seeking behavior at the philosophy faculties. While studies in information behavior draw on the cognitive viewpoint, information practice studies are guided by the ideas of social constructionism and collectivism \citep{Savolainen2007, Talja2005, Talja2007}. Humanities scholars information-seeking practices have also been addressed in several studies \citep{Benardou2013, Bulger2011, Given2018, Palmer2009}. In previous studies, however, the classic research process of humanities scholars has been examined, which is mainly based on literature research. Although the information-seeking in the humanities is also based on software tools, e.g. databases, web-based editions, search engines, or online journals \citep{Barrett2005, Rimmer2006}, the search for software itself is rarely discussed. One of these rare examples, however a non-humanities one, is Hepworths et al. \citeyearpar{Hepworth2017} study of journalism professors' information-seeking behavior. While seeking new online tools, journalism professors rely on other journalism professors, followed closely by media-related foundations, media professionals, and conferences.

%%======================================================================
%%      Theory
%%======================================================================
\chapter{Theory}
%%======================================================================

\section{Information Seeking}
- Information Science briefly described

	Seeking, Searching, Retrieval

- Information Seeking Research (Ingwersen2005)
-- Concepts: Strategies
-- Collaborative IS: Shah2013 

- Distinction between behaviour and practices: 

	the concepts of information behavior and information practice emerge from different discourses that open alternative viewpoints on information seeking. Savolainen2007 

	Bates2010 - information behaviour, Case2007 - information behaviour

- behaviour: wilson, ellis, kuhltau, Niedzwiedzka2003
-- different conceptualizations: intra/inter/extrapersonal (Feinman
-- transgender ib: pohjanen2016

- practices: McKenzie and Talja 

\section{Information Practices}
- Introduction:
	Savolainen2007, Talja2007

The social constructionist paradigm puts emphasis on social practices, ``the concrete and situated activities of interacting people, reproduced in routine social contexts across time and space’’ (Rosenbaum, 1993, p. 239). A focus on
practices rather than on behaviour shifts the analysis from cognitive to social and is consistent with the study of information seekers within their social context (for examples, see Rothbauer (2002), McKenzie and Davies (2002)).

- Starting with McKenzie
	McKenzie2003, 2003a

- and Talja

- further examples of Information Practices: Savolainen 2007 - LitReview

\section{Research Software}
- definition
- examples
- importance for research, in the research process

\subsection{Information Practices towards research software}
- examples of studies, what has been studied yet

\section{Domain Analysis: Humanities/Philology}
- short: humanities, long: philology
- definition
- characteristics: subjects, work procedures, tools, ...

\subsection{Information Practices of Humanists}
- examples of studies, what has been studied yet
- bisher nicht viel gefunden, practices of other scholars, but humanists seldom
	
%%======================================================================
%%      Research Design
%%======================================================================
\chapter{Research Design}
Since "\textit{[u]nderstanding the nature of information practices and their relation to the production of scholarship is important for both theoretical and applied work in library and information science (LIS)}" \citep[p. 165]{Palmer2009} this thesis sets out to study information practices of humanities scholars and their seeking for research software to better understand humanists needs and future LIS services \citep{Case2008, Cunningham2010}. With information practices we mean practices of seeking, managing, giving, and using information in context \citep{Palmer2009}. The aim of this work is to investigate the information-seeking practices of early-career philologists when seeking research software. This research focuses on information needs of philologists, their information sources, problems, contradictions and barriers in finding information and their information sharing about research software. Special emphasis will be placed on the respondents' recourse to their own research process and the knowledge and practice structures in their field \citep{Hjorland1995} which are socially constructed.\\

\begin{tabular}{p{2cm}p{12cm}}
\textbf{RQ1}: & What information seeking practices humanities scholars engage in when looking for software to use with research data? \\ 
\textbf{RQ2}: & How do domain specific structures shape the information practices of humanists? \\ 
\end{tabular}\\ 

I chose an exploratory study design \citep{Rinsdorf2013} where the personal realm of experience of each interviewee lies in the center of the analysis. Interviews are the main data gathering technique which are applied in a semi-structured manner, guided by interview guidelines, and implemented in a face-to-face manner \citep{Bryman2004} in German language. With the interviews I obtain emotions, thoughts, and intentions of the participants and discover their perspective of the social world \citep{Patton2002}. I will conduct 4-6 interviews of about 60-90 minutes length. The interview data will be anonmyized with amnesia\footnote{https://amnesia.openaire.eu} and analyzed with a qualitative content analysis to explore qualified hypotheses \citep{Kohlbacher2006, Krippendorff2012, Mayring2000, Mayring2014}. It enables the researcher to include textual information and to identify its properties systematically. I will make all data generated during the concept, survey, analysis, and writing phases publicly available on GitHub\footnote{\url{https://github.com/geyslein/Masters_Thesis}}, as long as it meets research ethics standards (e.g. interview audio records and unanonymized interview transcripts will be excluded).

\section{Data Gathering}

Interviews - zweistufig
- every interviewee interviewed twice
- due to corona
- no experience with videoconferencing: schneller erfahrungen sammeln, mehr interviews/längere dauer profitieren davon
- aufmerksamkeitsspanne geht schnell nach unten
- meine Präferenz aber auch die ersten beiden angefragten Personen fanden es in ordnung
- flexibilität: möglichkeit der analyse des ersten interviews und dann anpassungen möglich (grounded theory anteile)
- rekursion auf letztes interview möglich bei den interviewees, reflektion in der zwischenzeit 

\section{Data Processing}
- 

\section{Data Analysis}
The model/findings are derived from accounts of information seeking and not from observation of information seeking as it happened.

	
%%======================================================================
%%      Findings
%%======================================================================
	
\chapter{Findings}

%%======================================================================
%%      Discussion
%%======================================================================
\chapter{Discussion}

%%======================================================================
%%      Conclusion
%%======================================================================
\chapter{Conclusion}

%%======================================================================
%%      Zusammenfassung
%%======================================================================
\chapter{Zusammenfassung (German Conclusion)}

\bibliographystyle{apalike} 
%\bibliographystyle{plainnat} % use this to have URLs listed in References
\cleardoublepage			
\bibliography{../../Literatur/UB} % Path to your References.bib file		

%======================================================================
%	Selbstständigkeitserklärung
%======================================================================
\chapter*{Declaration of independence}
%Hiermit erkläre ich, dass ich die vorliegende Arbeit selbstständig angefertigt, nicht anderweitig zu Prfüngszwecken vorgelegt und keine anderen als die angegebenen Hilfsmittel verwendet habe. Sämtliche  wissentlich verwendeten Textausschnitte, Zitate oder Inhalte anderer Verfasser wurden ausdrücklich als solche gekennzeichnet.\\[2ex]
I hereby declare that I have prepared this paper independently, have not submitted it for testing purposes and have not used any other aids than those specified. All knowingly used text excerpts, quotations or contents of other authors have been explicitly marked as such. \\[2ex]

Leipzig,  \today\\[6ex]
\flushleft
\newlength\us
\settowidth{\us}{Ronny Gey}
\begin{tabular}{p{\us}}\hline
\centering\footnotesize Ronny Gey
\end{tabular}

%======================================================================
%	Anhang
%======================================================================
%\cleardoublepage\pdfbookmark[-1]{Anhang}{Anhang}
\cleardoublepage

\begin{appendices}
\appendixpage

\chapter{Validation of Interview Questions}

\cleardoublepage

\chapter{Interview Guide}
The contents...

\cleardoublepage

\chapter{Consent Form}

\end{appendices}


\end{document}